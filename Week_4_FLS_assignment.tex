% Options for packages loaded elsewhere
\PassOptionsToPackage{unicode}{hyperref}
\PassOptionsToPackage{hyphens}{url}
%
\documentclass[
]{article}
\usepackage{amsmath,amssymb}
\usepackage{lmodern}
\usepackage{iftex}
\ifPDFTeX
  \usepackage[T1]{fontenc}
  \usepackage[utf8]{inputenc}
  \usepackage{textcomp} % provide euro and other symbols
\else % if luatex or xetex
  \usepackage{unicode-math}
  \defaultfontfeatures{Scale=MatchLowercase}
  \defaultfontfeatures[\rmfamily]{Ligatures=TeX,Scale=1}
\fi
% Use upquote if available, for straight quotes in verbatim environments
\IfFileExists{upquote.sty}{\usepackage{upquote}}{}
\IfFileExists{microtype.sty}{% use microtype if available
  \usepackage[]{microtype}
  \UseMicrotypeSet[protrusion]{basicmath} % disable protrusion for tt fonts
}{}
\makeatletter
\@ifundefined{KOMAClassName}{% if non-KOMA class
  \IfFileExists{parskip.sty}{%
    \usepackage{parskip}
  }{% else
    \setlength{\parindent}{0pt}
    \setlength{\parskip}{6pt plus 2pt minus 1pt}}
}{% if KOMA class
  \KOMAoptions{parskip=half}}
\makeatother
\usepackage{xcolor}
\usepackage[margin=1in]{geometry}
\usepackage{color}
\usepackage{fancyvrb}
\newcommand{\VerbBar}{|}
\newcommand{\VERB}{\Verb[commandchars=\\\{\}]}
\DefineVerbatimEnvironment{Highlighting}{Verbatim}{commandchars=\\\{\}}
% Add ',fontsize=\small' for more characters per line
\usepackage{framed}
\definecolor{shadecolor}{RGB}{248,248,248}
\newenvironment{Shaded}{\begin{snugshade}}{\end{snugshade}}
\newcommand{\AlertTok}[1]{\textcolor[rgb]{0.94,0.16,0.16}{#1}}
\newcommand{\AnnotationTok}[1]{\textcolor[rgb]{0.56,0.35,0.01}{\textbf{\textit{#1}}}}
\newcommand{\AttributeTok}[1]{\textcolor[rgb]{0.77,0.63,0.00}{#1}}
\newcommand{\BaseNTok}[1]{\textcolor[rgb]{0.00,0.00,0.81}{#1}}
\newcommand{\BuiltInTok}[1]{#1}
\newcommand{\CharTok}[1]{\textcolor[rgb]{0.31,0.60,0.02}{#1}}
\newcommand{\CommentTok}[1]{\textcolor[rgb]{0.56,0.35,0.01}{\textit{#1}}}
\newcommand{\CommentVarTok}[1]{\textcolor[rgb]{0.56,0.35,0.01}{\textbf{\textit{#1}}}}
\newcommand{\ConstantTok}[1]{\textcolor[rgb]{0.00,0.00,0.00}{#1}}
\newcommand{\ControlFlowTok}[1]{\textcolor[rgb]{0.13,0.29,0.53}{\textbf{#1}}}
\newcommand{\DataTypeTok}[1]{\textcolor[rgb]{0.13,0.29,0.53}{#1}}
\newcommand{\DecValTok}[1]{\textcolor[rgb]{0.00,0.00,0.81}{#1}}
\newcommand{\DocumentationTok}[1]{\textcolor[rgb]{0.56,0.35,0.01}{\textbf{\textit{#1}}}}
\newcommand{\ErrorTok}[1]{\textcolor[rgb]{0.64,0.00,0.00}{\textbf{#1}}}
\newcommand{\ExtensionTok}[1]{#1}
\newcommand{\FloatTok}[1]{\textcolor[rgb]{0.00,0.00,0.81}{#1}}
\newcommand{\FunctionTok}[1]{\textcolor[rgb]{0.00,0.00,0.00}{#1}}
\newcommand{\ImportTok}[1]{#1}
\newcommand{\InformationTok}[1]{\textcolor[rgb]{0.56,0.35,0.01}{\textbf{\textit{#1}}}}
\newcommand{\KeywordTok}[1]{\textcolor[rgb]{0.13,0.29,0.53}{\textbf{#1}}}
\newcommand{\NormalTok}[1]{#1}
\newcommand{\OperatorTok}[1]{\textcolor[rgb]{0.81,0.36,0.00}{\textbf{#1}}}
\newcommand{\OtherTok}[1]{\textcolor[rgb]{0.56,0.35,0.01}{#1}}
\newcommand{\PreprocessorTok}[1]{\textcolor[rgb]{0.56,0.35,0.01}{\textit{#1}}}
\newcommand{\RegionMarkerTok}[1]{#1}
\newcommand{\SpecialCharTok}[1]{\textcolor[rgb]{0.00,0.00,0.00}{#1}}
\newcommand{\SpecialStringTok}[1]{\textcolor[rgb]{0.31,0.60,0.02}{#1}}
\newcommand{\StringTok}[1]{\textcolor[rgb]{0.31,0.60,0.02}{#1}}
\newcommand{\VariableTok}[1]{\textcolor[rgb]{0.00,0.00,0.00}{#1}}
\newcommand{\VerbatimStringTok}[1]{\textcolor[rgb]{0.31,0.60,0.02}{#1}}
\newcommand{\WarningTok}[1]{\textcolor[rgb]{0.56,0.35,0.01}{\textbf{\textit{#1}}}}
\usepackage{graphicx}
\makeatletter
\def\maxwidth{\ifdim\Gin@nat@width>\linewidth\linewidth\else\Gin@nat@width\fi}
\def\maxheight{\ifdim\Gin@nat@height>\textheight\textheight\else\Gin@nat@height\fi}
\makeatother
% Scale images if necessary, so that they will not overflow the page
% margins by default, and it is still possible to overwrite the defaults
% using explicit options in \includegraphics[width, height, ...]{}
\setkeys{Gin}{width=\maxwidth,height=\maxheight,keepaspectratio}
% Set default figure placement to htbp
\makeatletter
\def\fps@figure{htbp}
\makeatother
\setlength{\emergencystretch}{3em} % prevent overfull lines
\providecommand{\tightlist}{%
  \setlength{\itemsep}{0pt}\setlength{\parskip}{0pt}}
\setcounter{secnumdepth}{-\maxdimen} % remove section numbering
\ifLuaTeX
  \usepackage{selnolig}  % disable illegal ligatures
\fi
\IfFileExists{bookmark.sty}{\usepackage{bookmark}}{\usepackage{hyperref}}
\IfFileExists{xurl.sty}{\usepackage{xurl}}{} % add URL line breaks if available
\urlstyle{same} % disable monospaced font for URLs
\hypersetup{
  pdftitle={Week\_4\_FLS\_assignment},
  pdfauthor={Todd Garner},
  hidelinks,
  pdfcreator={LaTeX via pandoc}}

\title{Week\_4\_FLS\_assignment}
\author{Todd Garner}
\date{2023-01-21}

\begin{document}
\maketitle

{
\setcounter{tocdepth}{2}
\tableofcontents
}
\hypertarget{fls---week-4}{%
\section{FLS - Week 4}\label{fls---week-4}}

\hypertarget{fls-page-2---run-the-file-unit_4_web_scraping.r}{%
\subsection{FLS Page 2 - Run the file
``Unit\_4\_Web\_Scraping.R''}\label{fls-page-2---run-the-file-unit_4_web_scraping.r}}

Run the example code in the file Unit 4 Web Scraping.R This will help
tremendously with the web scraping in Part 1. You will have to learn the
grep and grepl functions which I think you will find very useful and we
will extend on in the future. (1 hour)

I opened the existing/provided file and copied and pasted into the R
code chunks below. Let's see how we do!

\#Live Session 4 For Live Session Web Scraping Code

\begin{Shaded}
\begin{Highlighting}[]
\CommentTok{\#install.packages("XML")}
\FunctionTok{library}\NormalTok{(XML) }\CommentTok{\#xml\_Parse}
\FunctionTok{library}\NormalTok{(dplyr)}
\end{Highlighting}
\end{Shaded}

\begin{verbatim}
## 
## Attaching package: 'dplyr'
\end{verbatim}

\begin{verbatim}
## The following objects are masked from 'package:stats':
## 
##     filter, lag
\end{verbatim}

\begin{verbatim}
## The following objects are masked from 'package:base':
## 
##     intersect, setdiff, setequal, union
\end{verbatim}

\begin{Shaded}
\begin{Highlighting}[]
\FunctionTok{library}\NormalTok{(tidyr)}
\FunctionTok{library}\NormalTok{(stringi)}
\FunctionTok{library}\NormalTok{(rvest) }\CommentTok{\#html\_table, html\_node}
\FunctionTok{library}\NormalTok{(ggplot2)}
\CommentTok{\#install.packages("RCurl")}
\FunctionTok{library}\NormalTok{(RCurl) }\CommentTok{\#getURL}
\end{Highlighting}
\end{Shaded}

\begin{verbatim}
## 
## Attaching package: 'RCurl'
\end{verbatim}

\begin{verbatim}
## The following object is masked from 'package:tidyr':
## 
##     complete
\end{verbatim}

\#Basics of Scraping XML

\hypertarget{method-1-xml}{%
\section{Method 1: XML}\label{method-1-xml}}

\begin{Shaded}
\begin{Highlighting}[]
\NormalTok{data }\OtherTok{\textless{}{-}}\FunctionTok{getURL}\NormalTok{(}\StringTok{"https://www.w3schools.com/xml/simple.xml"}\NormalTok{)}
\NormalTok{doc }\OtherTok{\textless{}{-}} \FunctionTok{xmlParse}\NormalTok{(data)}
\NormalTok{names }\OtherTok{\textless{}{-}} \FunctionTok{xpathSApply}\NormalTok{(doc,}\StringTok{"//name"}\NormalTok{,xmlValue)}
\NormalTok{price }\OtherTok{\textless{}{-}} \FunctionTok{xpathSApply}\NormalTok{(doc,}\StringTok{"//price"}\NormalTok{,xmlValue)}
\NormalTok{description }\OtherTok{\textless{}{-}} \FunctionTok{xpathSApply}\NormalTok{(doc,}\StringTok{"//description"}\NormalTok{,xmlValue)}
\NormalTok{bfasts }\OtherTok{=} \FunctionTok{data.frame}\NormalTok{(names,price,description)}
\NormalTok{bfasts}
\end{Highlighting}
\end{Shaded}

\begin{verbatim}
##                         names price
## 1             Belgian Waffles $5.95
## 2  Strawberry Belgian Waffles $7.95
## 3 Berry-Berry Belgian Waffles $8.95
## 4                French Toast $4.50
## 5         Homestyle Breakfast $6.95
##                                                                           description
## 1                   Two of our famous Belgian Waffles with plenty of real maple syrup
## 2                   Light Belgian waffles covered with strawberries and whipped cream
## 3 Light Belgian waffles covered with an assortment of fresh berries and whipped cream
## 4                                 Thick slices made from our homemade sourdough bread
## 5                 Two eggs, bacon or sausage, toast, and our ever-popular hash browns
\end{verbatim}

\begin{Shaded}
\begin{Highlighting}[]
\NormalTok{bfasts}\SpecialCharTok{$}\NormalTok{description}
\end{Highlighting}
\end{Shaded}

\begin{verbatim}
## [1] "Two of our famous Belgian Waffles with plenty of real maple syrup"                  
## [2] "Light Belgian waffles covered with strawberries and whipped cream"                  
## [3] "Light Belgian waffles covered with an assortment of fresh berries and whipped cream"
## [4] "Thick slices made from our homemade sourdough bread"                                
## [5] "Two eggs, bacon or sausage, toast, and our ever-popular hash browns"
\end{verbatim}

\begin{Shaded}
\begin{Highlighting}[]
\FunctionTok{length}\NormalTok{(}\FunctionTok{grep}\NormalTok{(}\StringTok{"covered"}\NormalTok{,bfasts}\SpecialCharTok{$}\NormalTok{description))}
\end{Highlighting}
\end{Shaded}

\begin{verbatim}
## [1] 2
\end{verbatim}

\begin{Shaded}
\begin{Highlighting}[]
\FunctionTok{grepl}\NormalTok{(}\StringTok{"covered"}\NormalTok{,bfasts}\SpecialCharTok{$}\NormalTok{description)}
\end{Highlighting}
\end{Shaded}

\begin{verbatim}
## [1] FALSE  TRUE  TRUE FALSE FALSE
\end{verbatim}

\begin{Shaded}
\begin{Highlighting}[]
\FunctionTok{sum}\NormalTok{(}\FunctionTok{grepl}\NormalTok{(}\StringTok{"covered"}\NormalTok{,bfasts}\SpecialCharTok{$}\NormalTok{description))}
\end{Highlighting}
\end{Shaded}

\begin{verbatim}
## [1] 2
\end{verbatim}

\begin{Shaded}
\begin{Highlighting}[]
\FunctionTok{which}\NormalTok{(}\FunctionTok{grepl}\NormalTok{(}\StringTok{"covered"}\NormalTok{,bfasts}\SpecialCharTok{$}\NormalTok{description))}
\end{Highlighting}
\end{Shaded}

\begin{verbatim}
## [1] 2 3
\end{verbatim}

\hypertarget{method-2-rvest}{%
\section{Method 2: rvest}\label{method-2-rvest}}

\begin{Shaded}
\begin{Highlighting}[]
\NormalTok{hp}\OtherTok{\textless{}{-}}\FunctionTok{read\_html}\NormalTok{(}\StringTok{"https://www.w3schools.com/xml/simple.xml"}\NormalTok{)}
\NormalTok{hp\_nameR }\OtherTok{\textless{}{-}} \FunctionTok{html\_nodes}\NormalTok{(hp,}\StringTok{"name"}\NormalTok{)}
\NormalTok{hp\_priceR }\OtherTok{\textless{}{-}} \FunctionTok{html\_nodes}\NormalTok{(hp,}\StringTok{"price"}\NormalTok{)}
\NormalTok{hp\_descR }\OtherTok{\textless{}{-}} \FunctionTok{html\_nodes}\NormalTok{(hp,}\StringTok{"description"}\NormalTok{)}
\NormalTok{hp\_nameR}
\end{Highlighting}
\end{Shaded}

\begin{verbatim}
## {xml_nodeset (5)}
## [1] <name>Belgian Waffles</name>
## [2] <name>Strawberry Belgian Waffles</name>
## [3] <name>Berry-Berry Belgian Waffles</name>
## [4] <name>French Toast</name>
## [5] <name>Homestyle Breakfast</name>
\end{verbatim}

\begin{Shaded}
\begin{Highlighting}[]
\NormalTok{hp\_name }\OtherTok{=} \FunctionTok{stri\_sub}\NormalTok{(hp\_nameR,}\DecValTok{7}\NormalTok{,}\SpecialCharTok{{-}}\DecValTok{8}\NormalTok{)}
\end{Highlighting}
\end{Shaded}

\begin{verbatim}
## Warning in stri_sub(hp_nameR, 7, -8): argument is not an atomic vector; coercing
\end{verbatim}

\begin{Shaded}
\begin{Highlighting}[]
\NormalTok{hp\_name}
\end{Highlighting}
\end{Shaded}

\begin{verbatim}
## [1] "Belgian Waffles"             "Strawberry Belgian Waffles" 
## [3] "Berry-Berry Belgian Waffles" "French Toast"               
## [5] "Homestyle Breakfast"
\end{verbatim}

\begin{Shaded}
\begin{Highlighting}[]
\NormalTok{hp\_price }\OtherTok{=} \FunctionTok{stri\_sub}\NormalTok{(hp\_priceR,}\DecValTok{8}\NormalTok{,}\SpecialCharTok{{-}}\DecValTok{9}\NormalTok{)}
\end{Highlighting}
\end{Shaded}

\begin{verbatim}
## Warning in stri_sub(hp_priceR, 8, -9): argument is not an atomic vector;
## coercing
\end{verbatim}

\begin{Shaded}
\begin{Highlighting}[]
\NormalTok{hp\_price}
\end{Highlighting}
\end{Shaded}

\begin{verbatim}
## [1] "$5.95" "$7.95" "$8.95" "$4.50" "$6.95"
\end{verbatim}

\begin{Shaded}
\begin{Highlighting}[]
\NormalTok{hp\_desc }\OtherTok{=} \FunctionTok{stri\_sub}\NormalTok{(hp\_descR,}\DecValTok{14}\NormalTok{,}\SpecialCharTok{{-}}\DecValTok{15}\NormalTok{)}
\end{Highlighting}
\end{Shaded}

\begin{verbatim}
## Warning in stri_sub(hp_descR, 14, -15): argument is not an atomic vector;
## coercing
\end{verbatim}

\begin{Shaded}
\begin{Highlighting}[]
\NormalTok{hp\_desc}
\end{Highlighting}
\end{Shaded}

\begin{verbatim}
## [1] "Two of our famous Belgian Waffles with plenty of real maple syrup"                  
## [2] "Light Belgian waffles covered with strawberries and whipped cream"                  
## [3] "Light Belgian waffles covered with an assortment of fresh berries and whipped cream"
## [4] "Thick slices made from our homemade sourdough bread"                                
## [5] "Two eggs, bacon or sausage, toast, and our ever-popular hash browns"
\end{verbatim}

\begin{Shaded}
\begin{Highlighting}[]
\NormalTok{bfast }\OtherTok{=} \FunctionTok{data.frame}\NormalTok{(hp\_name,hp\_price,hp\_desc)}
\FunctionTok{grep}\NormalTok{(}\StringTok{"toast"}\NormalTok{, bfast}\SpecialCharTok{$}\NormalTok{hp\_desc)}
\end{Highlighting}
\end{Shaded}

\begin{verbatim}
## [1] 5
\end{verbatim}

\begin{Shaded}
\begin{Highlighting}[]
\FunctionTok{grepl}\NormalTok{(}\StringTok{"toast"}\NormalTok{,bfast}\SpecialCharTok{$}\NormalTok{hp\_desc)}
\end{Highlighting}
\end{Shaded}

\begin{verbatim}
## [1] FALSE FALSE FALSE FALSE  TRUE
\end{verbatim}

\begin{Shaded}
\begin{Highlighting}[]
\FunctionTok{sum}\NormalTok{(}\FunctionTok{grepl}\NormalTok{(}\StringTok{"toast"}\NormalTok{,bfast}\SpecialCharTok{$}\NormalTok{hp\_desc))}
\end{Highlighting}
\end{Shaded}

\begin{verbatim}
## [1] 1
\end{verbatim}

\hypertarget{part-1-restaurant-data-from-baltimore}{%
\section{Part 1: Restaurant Data from
Baltimore!}\label{part-1-restaurant-data-from-baltimore}}

\textbf{Note: one of the methods learned in the last unit will work with
the the improper XML code that the URL below yields and the other will
create an error. Use one to get the job done.}

\emph{You have been hired by a restaurateur to do some research on Sushi
restaurants in Baltimore.} \emph{You have come across data on the web
contained in the following XML file.}\\
\emph{Data:
\url{https://d396qusza40orc.cloudfront.net/getdata\%2Fdata\%2Frestaurants.xml}}

\begin{verbatim}
1. Scrape the XML page for name, zipcode and city council district.  (Use either the XML or rvest package.) <
>>>>>>>>>DONE
2. Make a dataframe with just those columns.  
>>>>>>>>>DONE
3. Are there any Sushi restaurants in Baltimore? (Where the dataset is from.)
  a. If so, can you estimate how many?
\end{verbatim}

To solve question 3, what makes sense is to search the names column we
just created for the word ``sushi.'' We should use ``ignore.case =
TRUE'' and count the number of times the word sushi appears. We can
study the data results.

\begin{quote}
\begin{quote}
\begin{quote}
\begin{quote}
\begin{quote}
\begin{quote}
\begin{quote}
\begin{quote}
There appears to be 9 occurences of the word ``sushi'' in the names
column. Perhaps we should search for other possible sushi restaurant
names. I appear to have discovered a very cool way of displaying text.
The right pointing ``pointer'' repeated reveals interesting results as
seen in the knitted .html file result. Candidly, that's hard to believe
that there are only 9 sushi restaurants in Baltimore. I think I need to
dig deeper\ldots.somehow.
\end{quote}
\end{quote}
\end{quote}
\end{quote}
\end{quote}
\end{quote}
\end{quote}
\end{quote}

I substituted alternative names for ``sushi'' like tempura, hibachi,
hibatchi, sashimi, to no avail. I went to Google and searched ``How many
sushi restaurants are there in Baltimore.'' My first result was, ``Top
10 sushi restaurants in Baltimore.'' If there was a category
``Description'' I think this would make all the difference in the world.
But, with the data available, it appears that there are 9 with the word
sushi in the restaurant name. This is altogether unsatisfying as I'm
virtually certain there are more. I'm going to add neighborhood to my
results in a separate column. Chinatown may or may not be an indicator
and at this point I'm not sure how it would be utilized.

At that point, it makes sense to count the number of occurrences of
districts. This ``may'' be telling in that sushi may be found in
``restaurant row'' or in certain Asian neighborhoods or near certain
businesses like ``high tech'' or the like. Let's find out! Result: Not
much help.

\begin{Shaded}
\begin{Highlighting}[]
\CommentTok{\#   1. Scrape the XML page for name, zipcode and city council district.  (Use either the XML or rvest package.)}

\NormalTok{data }\OtherTok{\textless{}{-}}\FunctionTok{getURL}\NormalTok{(}\StringTok{"https://d396qusza40orc.cloudfront.net/getdata\%2Fdata\%2Frestaurants.xml"}\NormalTok{)}
\NormalTok{doc }\OtherTok{\textless{}{-}} \FunctionTok{xmlParse}\NormalTok{(data)}
\CommentTok{\#doc}
\NormalTok{names }\OtherTok{\textless{}{-}} \FunctionTok{xpathSApply}\NormalTok{(doc,}\StringTok{"//name"}\NormalTok{,xmlValue)}
\NormalTok{zipcode }\OtherTok{\textless{}{-}} \FunctionTok{xpathSApply}\NormalTok{(doc,}\StringTok{"//zipcode"}\NormalTok{,xmlValue)}
\NormalTok{councildistrict }\OtherTok{\textless{}{-}} \FunctionTok{xpathSApply}\NormalTok{(doc,}\StringTok{"//councildistrict"}\NormalTok{,xmlValue)}
\NormalTok{neighborhood }\OtherTok{\textless{}{-}} \FunctionTok{xpathSApply}\NormalTok{(doc,}\StringTok{"//neighborhood"}\NormalTok{,xmlValue)}
\NormalTok{area }\OtherTok{=} \FunctionTok{data.frame}\NormalTok{(names,zipcode,councildistrict)}
\FunctionTok{head}\NormalTok{(area)}
\end{Highlighting}
\end{Shaded}

\begin{verbatim}
##                   names zipcode councildistrict
## 1                   410   21206               2
## 2                  1919   21231               1
## 3                 SAUTE   21224               1
## 4    #1 CHINESE KITCHEN   21211              14
## 5 #1 chinese restaurant   21223               9
## 6             19TH HOLE   21218              14
\end{verbatim}

\begin{Shaded}
\begin{Highlighting}[]
\FunctionTok{head}\NormalTok{(area}\SpecialCharTok{$}\NormalTok{names)}
\end{Highlighting}
\end{Shaded}

\begin{verbatim}
## [1] "410"                   "1919"                  "SAUTE"                
## [4] "#1 CHINESE KITCHEN"    "#1 chinese restaurant" "19TH HOLE"
\end{verbatim}

\begin{Shaded}
\begin{Highlighting}[]
\CommentTok{\#area$neighborhood}
\FunctionTok{length}\NormalTok{(}\FunctionTok{grep}\NormalTok{(}\StringTok{"sushi"}\NormalTok{,area}\SpecialCharTok{$}\NormalTok{names, }\AttributeTok{ignore.case =} \ConstantTok{TRUE}\NormalTok{))}
\end{Highlighting}
\end{Shaded}

\begin{verbatim}
## [1] 9
\end{verbatim}

\begin{Shaded}
\begin{Highlighting}[]
\FunctionTok{length}\NormalTok{(}\FunctionTok{grep}\NormalTok{(}\StringTok{"sashimi"}\NormalTok{,area}\SpecialCharTok{$}\NormalTok{names, }\AttributeTok{ignore.case =} \ConstantTok{TRUE}\NormalTok{))}
\end{Highlighting}
\end{Shaded}

\begin{verbatim}
## [1] 0
\end{verbatim}

\begin{Shaded}
\begin{Highlighting}[]
\FunctionTok{length}\NormalTok{(}\FunctionTok{grep}\NormalTok{(}\StringTok{"asian"}\NormalTok{,area}\SpecialCharTok{$}\NormalTok{neighborhood, }\AttributeTok{ignore.case =} \ConstantTok{TRUE}\NormalTok{))}
\end{Highlighting}
\end{Shaded}

\begin{verbatim}
## [1] 0
\end{verbatim}

\begin{Shaded}
\begin{Highlighting}[]
\CommentTok{\#grepl("sushi",area$names)}
\FunctionTok{sum}\NormalTok{(}\FunctionTok{grepl}\NormalTok{(}\StringTok{"sushi"}\NormalTok{,area}\SpecialCharTok{$}\NormalTok{names))}
\end{Highlighting}
\end{Shaded}

\begin{verbatim}
## [1] 0
\end{verbatim}

\begin{Shaded}
\begin{Highlighting}[]
\FunctionTok{which}\NormalTok{(}\FunctionTok{grepl}\NormalTok{(}\StringTok{"sushi"}\NormalTok{,area}\SpecialCharTok{$}\NormalTok{names))}
\end{Highlighting}
\end{Shaded}

\begin{verbatim}
## integer(0)
\end{verbatim}

\begin{Shaded}
\begin{Highlighting}[]
\FunctionTok{head}\NormalTok{(area}\SpecialCharTok{$}\NormalTok{names)}
\end{Highlighting}
\end{Shaded}

\begin{verbatim}
## [1] "410"                   "1919"                  "SAUTE"                
## [4] "#1 CHINESE KITCHEN"    "#1 chinese restaurant" "19TH HOLE"
\end{verbatim}

\begin{Shaded}
\begin{Highlighting}[]
\FunctionTok{head}\NormalTok{(area}\SpecialCharTok{$}\NormalTok{neighborhood)}
\end{Highlighting}
\end{Shaded}

\begin{verbatim}
## NULL
\end{verbatim}

\begin{Shaded}
\begin{Highlighting}[]
\FunctionTok{length}\NormalTok{(}\FunctionTok{grep}\NormalTok{(}\StringTok{"sushi"}\NormalTok{,area}\SpecialCharTok{$}\NormalTok{names, }\AttributeTok{ignore.case =} \ConstantTok{TRUE}\NormalTok{))}
\end{Highlighting}
\end{Shaded}

\begin{verbatim}
## [1] 9
\end{verbatim}

\begin{Shaded}
\begin{Highlighting}[]
\FunctionTok{length}\NormalTok{(}\FunctionTok{grep}\NormalTok{(}\StringTok{"sashimi"}\NormalTok{,area}\SpecialCharTok{$}\NormalTok{names, }\AttributeTok{ignore.case =} \ConstantTok{TRUE}\NormalTok{))}
\end{Highlighting}
\end{Shaded}

\begin{verbatim}
## [1] 0
\end{verbatim}

\begin{Shaded}
\begin{Highlighting}[]
\FunctionTok{length}\NormalTok{(}\FunctionTok{grep}\NormalTok{(}\StringTok{"asian"}\NormalTok{,area}\SpecialCharTok{$}\NormalTok{neighborhood, }\AttributeTok{ignore.case =} \ConstantTok{TRUE}\NormalTok{))}
\end{Highlighting}
\end{Shaded}

\begin{verbatim}
## [1] 0
\end{verbatim}

\begin{Shaded}
\begin{Highlighting}[]
\FunctionTok{head}\NormalTok{(}\FunctionTok{grepl}\NormalTok{(}\StringTok{"sushi"}\NormalTok{,area}\SpecialCharTok{$}\NormalTok{names))}
\end{Highlighting}
\end{Shaded}

\begin{verbatim}
## [1] FALSE FALSE FALSE FALSE FALSE FALSE
\end{verbatim}

\begin{Shaded}
\begin{Highlighting}[]
\FunctionTok{sum}\NormalTok{(}\FunctionTok{grepl}\NormalTok{(}\StringTok{"sushi"}\NormalTok{,area}\SpecialCharTok{$}\NormalTok{names))}
\end{Highlighting}
\end{Shaded}

\begin{verbatim}
## [1] 0
\end{verbatim}

\begin{Shaded}
\begin{Highlighting}[]
\FunctionTok{which}\NormalTok{(}\FunctionTok{grepl}\NormalTok{(}\StringTok{"sushi"}\NormalTok{,area}\SpecialCharTok{$}\NormalTok{names))}
\end{Highlighting}
\end{Shaded}

\begin{verbatim}
## integer(0)
\end{verbatim}

\hypertarget{filter-the-dataframe-for-just-downtown-restaurants-council-district-11.}{%
\subsection{4. Filter the dataframe for just downtown restaurants
(Council District
11).}\label{filter-the-dataframe-for-just-downtown-restaurants-council-district-11.}}

\begin{Shaded}
\begin{Highlighting}[]
\NormalTok{area\_11 }\OtherTok{\textless{}{-}}\NormalTok{ area[area}\SpecialCharTok{$}\NormalTok{councildistrict }\SpecialCharTok{==} \StringTok{"11"}\NormalTok{,] }
\FunctionTok{head}\NormalTok{(area\_11)}
\end{Highlighting}
\end{Shaded}

\begin{verbatim}
##                     names zipcode councildistrict
## 15       A TASTE OF CHINA   21202              11
## 16  ABACROMBIE FINE FOODS   21201              11
## 22       AKBAR RESTAURANT   21201              11
## 29      BAY ATLANTIC CLUB   21212              11
## 33                BEDROCK   21201              11
## 34 BEDROCK (KITCHEN AREA)   21201              11
\end{verbatim}

There are 277 total restaurants in District 11.

\hypertarget{are-there-any-sushi-restaurants-downtown-research-the-grep-function}{%
\subsection{5. Are there any Sushi restaurants downtown? \# research the
``grep''
function}\label{are-there-any-sushi-restaurants-downtown-research-the-grep-function}}

\begin{Shaded}
\begin{Highlighting}[]
\FunctionTok{length}\NormalTok{(}\FunctionTok{grep}\NormalTok{(}\StringTok{"sushi"}\NormalTok{,area\_11, }\AttributeTok{ignore.case =} \ConstantTok{TRUE}\NormalTok{))}
\end{Highlighting}
\end{Shaded}

\begin{verbatim}
## [1] 1
\end{verbatim}

\begin{quote}
\begin{quote}
\begin{quote}
\begin{quote}
\begin{quote}
\begin{quote}
\begin{quote}
\begin{quote}
\begin{quote}
\begin{quote}
\begin{quote}
\begin{quote}
There appears to be one with the word sushi in the restaurant name. Once
again, it's unimaginable to me that there is only one sushi restaurant
in downtown Baltimore. If there was a description of the business, I
believe the search would be much more fruitful.
\end{quote}
\end{quote}
\end{quote}
\end{quote}
\end{quote}
\end{quote}
\end{quote}
\end{quote}
\end{quote}
\end{quote}
\end{quote}
\end{quote}

\hypertarget{if-so-estimate-how-many-sushi-restaurants-are-in-downtown}{%
\subsection{6. If so, estimate how many ``Sushi'' restaurants are in
Downtown}\label{if-so-estimate-how-many-sushi-restaurants-are-in-downtown}}

\begin{quote}
\begin{quote}
\begin{quote}
\begin{quote}
\begin{quote}
\begin{quote}
\begin{quote}
\begin{quote}
\begin{quote}
\begin{quote}
\begin{quote}
\begin{quote}
From the data available, it would appear that there is only one sushi
restaurant in Downtown Baltimore.
\end{quote}
\end{quote}
\end{quote}
\end{quote}
\end{quote}
\end{quote}
\end{quote}
\end{quote}
\end{quote}
\end{quote}
\end{quote}
\end{quote}

\hypertarget{make-a-barplot-of-the-estimated-number-of-restaurants-sushi-or-otherwise-in-each-council.}{%
\subsection{7. Make a barplot of the estimated number of restaurants
(Sushi or otherwise) in each
council.}\label{make-a-barplot-of-the-estimated-number-of-restaurants-sushi-or-otherwise-in-each-council.}}

\begin{Shaded}
\begin{Highlighting}[]
\FunctionTok{unique}\NormalTok{(area}\SpecialCharTok{$}\NormalTok{councildistrict)}
\end{Highlighting}
\end{Shaded}

\begin{verbatim}
##  [1] "2"  "1"  "14" "9"  "13" "7"  "10" "5"  "11" "6"  "12" "3"  "4"  "8"
\end{verbatim}

There appears to be 13 districts. I need to get the names of the
restaurants AND the individual districts and then length (to count) and
then once defined, use ggplot to plot a bar graph showing the number of
restaurants in each district.

\begin{Shaded}
\begin{Highlighting}[]
\NormalTok{list }\OtherTok{\textless{}{-}} \FunctionTok{c}\NormalTok{(}\DecValTok{1}\NormalTok{, }\DecValTok{2}\NormalTok{, }\DecValTok{3}\NormalTok{, }\DecValTok{4}\NormalTok{, }\DecValTok{5}\NormalTok{, }\DecValTok{6}\NormalTok{, }\DecValTok{7}\NormalTok{, }\DecValTok{8}\NormalTok{, }\DecValTok{9}\NormalTok{, }\DecValTok{10}\NormalTok{, }\DecValTok{11}\NormalTok{, }\DecValTok{12}\NormalTok{, }\DecValTok{13}\NormalTok{)}
\CommentTok{\#final\_sushi \textless{} list}
\CommentTok{\#Dist\_sushi \textless{}{-} data.frame(area$councildistrict)}
\CommentTok{\#Dist\_sushi \textless{}{-} data.frame(Dist\_sushi)}
\ControlFlowTok{for}\NormalTok{ (list }\ControlFlowTok{in} \DecValTok{1}\SpecialCharTok{:}\FunctionTok{length}\NormalTok{(list)) \{}
\CommentTok{\#grep(list, area$councildistrict == list)}
\NormalTok{District\_sushi }\OtherTok{\textless{}{-}} \FunctionTok{length}\NormalTok{(}\FunctionTok{grep}\NormalTok{(}\StringTok{"sushi"}\NormalTok{,area}\SpecialCharTok{$}\NormalTok{names, }\AttributeTok{ignore.case =} \ConstantTok{TRUE}\NormalTok{, }\AttributeTok{value =} \ConstantTok{FALSE}\NormalTok{))}
\CommentTok{\#append(Dist\_sushi, District\_sushi)}
\FunctionTok{print}\NormalTok{(}\FunctionTok{paste}\NormalTok{(District\_sushi))}
\CommentTok{\#append(District\_sushi, area$councildistrict)}
\NormalTok{\}}
\end{Highlighting}
\end{Shaded}

\begin{verbatim}
## [1] "9"
## [1] "9"
## [1] "9"
## [1] "9"
## [1] "9"
## [1] "9"
## [1] "9"
## [1] "9"
## [1] "9"
## [1] "9"
## [1] "9"
## [1] "9"
## [1] "9"
\end{verbatim}

\begin{Shaded}
\begin{Highlighting}[]
\FunctionTok{head}\NormalTok{(District\_sushi)}
\end{Highlighting}
\end{Shaded}

\begin{verbatim}
## [1] 9
\end{verbatim}

\begin{Shaded}
\begin{Highlighting}[]
\CommentTok{\#area\_11 \textless{}{-} area[area$councildistrict == "11",] }
\end{Highlighting}
\end{Shaded}

These names do not have the word sushi in them, but I'll keep working on
it to refine it. I must admit I'm stumped by this one. I need a match to
a name in one column, based on the value in another column. Seems easy
enough, but here we are. I feel like I should know this.

\end{document}
